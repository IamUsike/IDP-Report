% Chapter 6 content - save this as chapter6-conclusion.tex

\chapter{Conclusion}

This project was designed to address a critical issue in Indian agriculture: the unpredictability of monsoon patterns and their direct impact on the pricing of crops like soybean and onion.The approach aimed at improving accuracy in price forecasting using satellite images, rainfall, and historic market prices. The study was characterized by three interlinked goals: (1) to obtain and preprocess the crop- and monsoon-specific data of Karnataka; (2) to develop machine learning models that can predict the price using environmental and historical variables; and (3) to develop an interactive web-based platform that conveys such predictions to farmers and policymakers through dynamic visualizations.

In order to achieve these goals, the market records with more than 1,000 records over a period of 5 7 years were collected and cleaned. Machine learning models, such as Random Forest, SVR and Gradient Boosting were applied and tuned with GridSearchCV. Input variables were weather information, NDVI indexes and market records in the training process. The web application was developed in React.js, Tailwind CSS, and Node.js as frontend and backend services and the data processes were assigned to MongoDB and PostgreSQL. The visualisation was incorporated with the help of Recharts and Chart.js, thus, offering end-users the conveniently obtained insights based on the result of the modelling stage. This end to end pipeline helped the project team to incorporate data processing, modelling and deployment in an efficient and coherent way.

The outcomes reflect successful implementation of the objectives, the predictive ability of a machine-learning framework to forecast agricultural commodity prices. In particular, the analysis covers soybean and onion markets, and the former market is endowed with a 95\% price forecasting error rate and a Mean Absolute Error (MAE) of \textbf{Rs. 353} when computed based on a current market quotation benchmark. The estimation is achieved by combining machine-learning model with climatic covariates, i.e., rainfall data and normalized difference vegetation index (NDVI) anomaly. Once these climatic variables are included, model performance is significantly higher and average predictive accuracy is increased by about 40\% compared to baseline methods. All these findings are indicative of the fact that a unified data-driven price-modelling platform, which has the ability to operate in real-time, can potentially provide high-quality decision support to the agricultural stakeholders in India during the monsoon season.


\section{Future Scope}
The model in consideration has sufficient functionality in its current scope but there are a number of limitations which need to be considered. First, it is limited in its scope to Karnataka and Kharif crops, that is, onion and soybean, hence, it cannot be used outside of these settings; increased applicability would require further data incorporation and retraining. Second, the model currently fails to consider economic factors like government subsidy, transportation costs, or market dynamics caused by policy changes, among others, which can have a considerable impact on the accuracy of price forecasts.

In future, the framework may be extended to support more varieties of crops, states and languages. It may be enhanced with real-time weather stations based on IoT and mobile systems to enable field-level predictions, and additional adaptations may comprise profitability estimates, the intelligent selection of crops based on soil and precipitation predictions, and claims insurance checks. Additionally, the research of deep-learning architectures, namely BiLSTM and CNN-LSTM, may significantly increase the accuracy of predicting.

\section{Learning Outcomes of the Project}
\begin{itemize}
    \item Gained practical knowledge of machine learning algorithms and applied to real-world agricultural data.
    \item Developed expertise in full-stack development using React.js, Tailwind CSS, Node.js, and PostgreSQL/MongoDB for building scalable web applications.
    \item Learned advanced data preprocessing techniques including feature engineering, hyperparameter tuning, and cross-validation using scikit-learn.
    \item Enhanced ability to visualize complex data through libraries like Recharts and Chart.js for better communication.
    \item Understood the social and economic implications of agricultural technology for farmers and policy-level decision-making.
\end{itemize}
