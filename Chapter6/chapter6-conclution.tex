% Chapter 6 content - save this as chapter6-conclusion.tex

\chapter{Conclusion}

This project addresses the rising problem of erratic monsoon conditions affecting Indian agriculture by proposing a system that calculates crop losses and forecasts the prices of weather-sensitive crops such as onion and soybean in the market. The first objective was the acquisition and pre-processing of data, including satellite NDVI data, long-term rainfall patterns, and crop prices across more than 1000 markets. The second objective was analysis and modeling that focused on defining correlations between weather conditions and crop performance using machine learning models. The third objective was prediction and validation to develop a tool that provides timely and location-specific information to farmers, policymakers, and insurers.

To achieve these objectives, we constructed a modular pipeline that cleans and processes multi-seasonal data, performs forecasting using Random Forest regression models, and incorporates monsoon-related variables in the forecasting pipeline. The software has been deployed as a web application using React, Tailwind CSS, Node.js, and MongoDB/PostgreSQL. Due to its scalable and user-friendly design, the system enables access to visualizations with Recharts and Chart.js, filtering results by crop or region, simulating future scenarios, and receiving real-time notifications.

The results for each objective were significant. The integrated database covered a wide range of crops (500+) and markets (1,000+). The prediction model demonstrated the capability to predict prices with an accuracy of up to 95\% and a mean absolute error (MAE) of 353 for both onion and soybean. The platform gained the ability to identify risk areas and indicators of crop failure reliably through the incorporation of NDVI anomalies and rainfall trends. The transformation from reactive estimation of crop losses to proactive and environment-sensitive forecasting provides a comprehensive decision support tool for enhancing agricultural planning and minimizing risks.

\section{Future Scope}

Although the existing system can efficiently combine NDVI data, rainfall patterns, and historical prices of crops such as onion and soybean, there are certain limitations. The system is trained on limited crops and cannot be applied on a large scale to other crops without additional training. Moreover, satellite data granularity and real-time availability might constrain spatial accuracy, particularly for small agricultural fields. There was also a lack of validation against ground-truth losses or farmer-reported yield surveys due to resource constraints.

The system can be expanded in the future to include other types of crops and localities, particularly those that are underrepresented due to different agro-climatic conditions. It can be integrated with IoT-based soil sensors and weather sensors to enhance real-time prediction accuracy. Additionally, the platform may be extended to offer dynamic sowing/harvest recommendations, automated insurance claims processing, and blockchain-based crop sales records for verification purposes. Another way to enhance accessibility for farmers in India is to ensure that the mobile version supports multiple languages.

\section{Learning Outcomes of the Project}

The key learning outcomes from this project include:

\begin{itemize}
	\item Gained practical knowledge in applying machine learning models to real agricultural data
	\item Acquired experience in preprocessing and analyzing data from various sources, including NDVI time-series, rainfall data, and market price records
	\item Developed full-stack web development skills using technologies like React, Node.js, Tailwind CSS, and MongoDB/PostgreSQL
	\item Enhanced data visualization skills for presenting agricultural and pricing patterns using libraries such as Recharts and Chart.js
	\item Understood the complexities of integrating multi-source data for agricultural decision support systems
	\item Learned the importance of user-centric design in developing tools for diverse stakeholders in agriculture
\end{itemize}